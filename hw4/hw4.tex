\documentclass[12pt,a4paper,reqno,parskip=full]{amsart}
\usepackage{amsmath}
\usepackage{amsfonts}
\usepackage{amssymb}
\usepackage{parskip}

\begingroup
\makeatletter
\@for\theoremstyle:=definition,remark,plain\do{
\expandafter\g@addto@macro\csname th@\theoremstyle\endcsname{
\addtolength\thm@preskip\parskip}}
\endgroup

\numberwithin{equation}{section}
\addtolength{\textwidth}{3 truecm}
\addtolength{\textheight}{1 truecm}
\setlength{\voffset}{-.6 truecm}
\setlength{\hoffset}{-1.3 truecm}
\theoremstyle{plain}

\newtheorem{theorem}[subsection]{Theorem}
\newtheorem{proposition}[subsection]{Proposition}
\newtheorem{lemma}[subsection]{Lemma}
\newtheorem{corollary}[subsection]{Corollary}
\newtheorem{claim}[subsection]{Claim}
\newtheorem{conjecture}[subsection]{Conjecture}
\newtheorem{question}[subsection]{Question}
\newtheorem{remark}[subsection]{Remark}
\newtheorem{definition}[subsection]{Definition}
\newtheorem{example}[subsection]{Example}

\theoremstyle{definition}

\renewcommand{\leq}{\leqslant}
\renewcommand{\geq}{\geqslant}
\newcommand{\eps}{\varepsilon}

\DeclareMathOperator{\Aut}{Aut}
\DeclareMathOperator{\modulo}{mod}
\DeclareMathOperator{\End}{End}
\DeclareMathOperator{\Id}{Id}

\def\AA{{\mathcal A}}
\def\CC{{\mathcal C}}
\def\DD{{\mathcal D}}
\def\E{{\mathbb E}}
\def\EE{{\mathcal E}}
\def\FF{{\mathbb F}}
\def\II{{\mathcal I}}
\def\N{{\mathbb N}}
\def\OO{{\mathcal O}}
\def\PP{{\mathcal P}}
\def\Q{{\mathbb Q}}
\def\R{{\mathbb R}}
\def\S{{\mathbb S}}
\def\SS{{\mathcal S}}
\def\UU{{\mathcal U}}
\def\Z{{\mathbb Z}}

\begin{document}

\title{Homework 4}

\author{E. Taber McFarlin}

\maketitle

I pledge my honor that I have abided by the Stevens Honor System

\begin{enumerate}
  \item Disprove the following statements.
        \begin{enumerate}
          \item For any $x,y\in\mathbb{R}$, it is true that $\log(xy)=\log(x)+\log(y)$.
                \begin{proof} Consider $x,y<0$

                  $xy\in\mathbb{N}$, therefore $\log(xy)\in\mathbb{R}$.

                  $log(x) + \log(y)\notin\mathbb{R}$, therefore the two numbers cannot be equal.
                \end{proof}

          \item For every $n\in\mathbb{Z}$, the integer $f(n)=n^2-n+11$ is prime.
                \begin{proof} Consider $f(50)$
                  \begin{align*}
                    f(50) & = 50^2 - 50 + 11 \\
                    f(50) & = 2461
                  \end{align*}
                  2461 is not a prime number, its prime factors are 23 and 107.
                \end{proof}
        \end{enumerate}

  \item A triple of integers $(a,b,c)$, where $a,b,c\in\mathbb{Z}$, is said to be a Pythagorean triple if $a^2+b^2=c^2$. Prove that if $(a,b,c)$ is a Pythagorean triple, then $a$ or $b$ is even.
        \begin{proof}
          Assume $a$ and $b$ are odd. Let $k,m,n\in\mathbb{Z}$, $a = 2k + 1$, $b = 2m + 1$
          \begin{align*}
            (2k + 1)^2 + (2m + 1)^2      & = c^2 \\
            4k^2 + 4m^2 + 4k + 4m + 2    & = c^2 \\
            2(2k^2 + 2m^2 + 2k + 2m + 1) & = c^2
          \end{align*}
          This shows that if $a$ and $b$ are odd, then $c^2$, and therefore $c$, must be even. Let $c = 2n$
          \begin{align*}
            (2k + 1)^2 + (2m + 1)^2    & = (2n)^2 \\
            4k^2 + 4m^2 + 4k + 4m + 2  & = 4n^2     \\
            4k^2 + 4m^2 + 4k + 4m + 4n^2 & = 2      \\
            4(k^2 + m^2 + n^2 + k + m)   & = 2
          \end{align*}
          This would show that 4 is a factor of 2 which is not true.
        \end{proof}

  \item Prove that the number $\log_2(5)$ is irrational.
        \begin{proof}
          Let $a,b\in\mathbb{Z}$
          \begin{align*}
            \frac{a}{b}     & =\log_2(5) \\
            2^{\frac{a}{b}} & = 5        \\
            2^a             & = 5^b
          \end{align*}
          This cannot be true because $2^a$ must be even and $5^b$ must be odd.
        \end{proof}

  \item Prove that there exist distinct irrational numbers $x$ and $y$ such that $x^y$ is rational.
        \begin{proof}
          If $\sqrt{2}^{\sqrt{2}}\notin\mathbb{Q}$ then $x = \sqrt{2}^{\sqrt{2}}$, $y = \sqrt{2}$ is a solution.
          \[
            \sqrt{2}^{\sqrt{2}^{\sqrt{2}}} = \sqrt{2}^{\sqrt{2}\sqrt{2}} = \sqrt{2}^2 = 2
          \]

          If $\sqrt{2}^{\sqrt{2}}\in\mathbb{Q}$ then $x = \sqrt{2}$, $y = 2\sqrt{2}$ is a valid solution.

          Let $a,b\in\mathbb{Z}$
          \begin{align*}
            \sqrt{2}^{\sqrt{2}}                & = \frac{a}{b}                \\
            \left(\sqrt{2}^{\sqrt{2}}\right)^2 & = \left(\frac{a}{b}\right)^2 \\
            \sqrt{2}^{2\sqrt{2}}               & = \frac{a^2}{b^2}
          \end{align*}
        \end{proof}
\end{enumerate}

\end{document}