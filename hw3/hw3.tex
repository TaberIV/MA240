\documentclass[12pt,a4paper,reqno,parskip=full]{amsart}
\usepackage{amsmath}
\usepackage{amsfonts}
\usepackage{amssymb}
\usepackage{parskip}

\begingroup
\makeatletter
\@for\theoremstyle:=definition,remark,plain\do{
\expandafter\g@addto@macro\csname th@\theoremstyle\endcsname{
\addtolength\thm@preskip\parskip}}
\endgroup

\numberwithin{equation}{section}
\addtolength{\textwidth}{3 truecm}
\addtolength{\textheight}{1 truecm}
\setlength{\voffset}{-.6 truecm}
\setlength{\hoffset}{-1.3 truecm}
\theoremstyle{plain}

\newtheorem{theorem}[subsection]{Theorem}
\newtheorem{proposition}[subsection]{Proposition}
\newtheorem{lemma}[subsection]{Lemma}
\newtheorem{corollary}[subsection]{Corollary}
\newtheorem{claim}[subsection]{Claim}
\newtheorem{conjecture}[subsection]{Conjecture}
\newtheorem{question}[subsection]{Question}
\newtheorem{remark}[subsection]{Remark}
\newtheorem{definition}[subsection]{Definition}
\newtheorem{example}[subsection]{Example}

\theoremstyle{definition}

\renewcommand{\leq}{\leqslant}
\renewcommand{\geq}{\geqslant}
\newcommand{\eps}{\varepsilon}

\DeclareMathOperator{\Aut}{Aut}
\DeclareMathOperator{\modulo}{mod}
\DeclareMathOperator{\End}{End}
\DeclareMathOperator{\Id}{Id}

\def\AA{{\mathcal A}}
\def\CC{{\mathcal C}}
\def\DD{{\mathcal D}}
\def\E{{\mathbb E}}
\def\EE{{\mathcal E}}
\def\FF{{\mathbb F}}
\def\II{{\mathcal I}}
\def\N{{\mathbb N}}
\def\OO{{\mathcal O}}
\def\PP{{\mathcal P}}
\def\Q{{\mathbb Q}}
\def\R{{\mathbb R}}
\def\S{{\mathbb S}}
\def\SS{{\mathcal S}}
\def\UU{{\mathcal U}}
\def\Z{{\mathbb Z}}

\begin{document}

\title{Homework 3}

\author{E. Taber McFarlin}

\maketitle

I pledge my honor that I have abided by the Stevens Honor System.

\renewcommand{\labelenumii}{\roman{enumii}}
\begin{enumerate}
  \item
        \begin{proposition}
          Let $x, y \in \mathbb{Z}$ Prove that if $xy$ is odd, then $x^2 + y^2$ is even.
        \end{proposition}

        \begin{proof}
          As shown below, the product $xy$ is odd if and only if $x$ and $y$ are both odd.

          Case 1: Assume, without loss of generality, that $x$ is even, then: \\
          Let $k\in\mathbb{Z}$
          \[xy = 2ky\]
          Case 2: $x$ and $y$ are both odd:
          \[(2k + 1)(2m + 1) = 2(2km + k + m) + 1.\]
          So $x$ and $y$ are both odd.

          The square of an odd number is odd, therefore both $x^2$ and $y^2$ are odd. It is also true that the sum of two odd numbers is even. Therefore, if $xy$ is odd, then $x^2 + y^2$ is even.
        \end{proof}

  \item
        \begin{proposition}
          Let $A=\{1,2,3\}$ and $B=\{2,3,4\}$ be subsets of $S=\{1,2,3,4\}$, and let $x\in S$.
          Prove that $2x^2-5x$ is either positive and even or negative and odd
          if and only if $x\not\in A\cap B$.
        \end{proposition}
        \begin{proof}
          There are two cases where $x \notin A\cap B$ ($x=1$ and $x=4$),
          and two cases where $x \in A\cap B$ ($x=2$ and $x=3$).

          \begin{enumerate}
            \item $x = 1$
                  \[2(1)^2 - 5(1) = -3\]
                  $-3$ is negative and odd.
            \item $x = 4$
                  \[2(4)^2 - 5(4) = 32 - 20 = 12\]
                  12 is positive and even.
            \item $x = 2$
                  \[2(2)^2 - 5(2) = 8 - 10 = -2\]
                  $-2$ is negative and even.
            \item $x = 3$
                  \[2(3)^2 - 5(3) = 18 - 15 = 3\]
                  3 is positive and odd.
          \end{enumerate}

          In all of the cases, and only the cases, where $x \notin A\cap B$ and $x \in S$, is
          $2x^2-5x$ either positive and even or negative and odd.
        \end{proof}

  \item
        \begin{proposition}
          Prove that the function $d(x,y) = \left|\log\left(\frac{y}{x}\right)\right|$
          is a metric on the set of positive real numbers.
        \end{proposition}
        \begin{proof} This proof requires showing the four conditions of a metric:
          \begin{enumerate}
            \item $d(x,y) \geq 0$
                  \[
                    \left|\log\left(\frac{y}{x}\right)\right| \geq 0
                  \]
            \item $d(x,y) = 0 \iff x = y$
                  \[
                    \log\left(\frac{x}{x}\right) = \log(1) = 0
                  \]
                  log is a continuous function that whose value is equal to 0 at only one point, when 1 is the input. $\frac{x}{y}$ can only be 1 when $x = y$.
            \item $d(x,y) = d(y,x)$
                  Given
                  \[\log(n) = -\log(n^{-1})\]

                  It follows that:
                  \begin{align*}
                    d(x,y) & = \left|\log\left(\frac{y}{x}\right)\right|                                                     \\
                    d(y,x) & = \left|\log\left((\frac{y}{x})^{-1}\right)\right| = \left|-\log\left(\frac{y}{x}\right)\right| \\
                    d(x,y) & = d(y,x).
                  \end{align*}
            \item{$d(x,z) \leq d(x,y) + d(y,z)$}
                  \begin{align*}
                    \left|\log\left(\frac{z}{x}\right)\right| & \leq \left|\log\left(\frac{y}{x}\right)\right| + \left|\log\left(\frac{z}{y}\right)\right| \\
                    |\log(z) - \log(x)|                       & \leq |\log(y) - \log(x)| + |\log(z) - \log(y)|                                             \\
                  \end{align*}
                  This inequality has the same form as the triangle inequality for real numbers where: $d(x, y) = |y - x|$
                  \[|z - x|\leq |y - x| + |z - y|\]
          \end{enumerate}
        \end{proof}
\end{enumerate}
\end{document}