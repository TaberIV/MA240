\documentclass[12pt,a4paper,reqno,parskip=full]{amsart}
\usepackage{amsmath}
\usepackage{amsfonts}
\usepackage{amssymb}
\usepackage{parskip}

\begingroup
\makeatletter
\@for\theoremstyle:=definition,remark,plain\do{
\expandafter\g@addto@macro\csname th@\theoremstyle\endcsname{
\addtolength\thm@preskip\parskip}}
\endgroup

\numberwithin{equation}{section}
\addtolength{\textwidth}{3 truecm}
\addtolength{\textheight}{1 truecm}
\setlength{\voffset}{-.6 truecm}
\setlength{\hoffset}{-1.3 truecm}
\theoremstyle{plain}

\newtheorem{theorem}[subsection]{Theorem}
\newtheorem{proposition}[subsection]{Proposition}
\newtheorem{lemma}[subsection]{Lemma}
\newtheorem{corollary}[subsection]{Corollary}
\newtheorem{claim}[subsection]{Claim}
\newtheorem{conjecture}[subsection]{Conjecture}
\newtheorem{question}[subsection]{Question}
\newtheorem{remark}[subsection]{Remark}
\newtheorem{definition}[subsection]{Definition}
\newtheorem{example}[subsection]{Example}

\theoremstyle{definition}

\renewcommand{\leq}{\leqslant}
\renewcommand{\geq}{\geqslant}
\newcommand{\eps}{\varepsilon}

\DeclareMathOperator{\Aut}{Aut}
\DeclareMathOperator{\modulo}{mod}
\DeclareMathOperator{\End}{End}
\DeclareMathOperator{\Id}{Id}

\def\AA{{\mathcal A}}
\def\CC{{\mathcal C}}
\def\DD{{\mathcal D}}
\def\E{{\mathbb E}}
\def\EE{{\mathcal E}}
\def\FF{{\mathbb F}}
\def\II{{\mathcal I}}
\def\N{{\mathbb N}}
\def\OO{{\mathcal O}}
\def\PP{{\mathcal P}}
\def\Q{{\mathbb Q}}
\def\R{{\mathbb R}}
\def\S{{\mathbb S}}
\def\SS{{\mathcal S}}
\def\UU{{\mathcal U}}
\def\Z{{\mathbb Z}}

\begin{document}

\title{Homework 5}

\author{E. Taber McFarlin}

\maketitle

\begin{enumerate}
  \item Prove that $\displaystyle1^2+2^2+\ldots+n^2=\frac{n(n+1)(2n+1)}{6}$ for all $n\in\mathbb{N}$.
        \begin{proof} Base step:
          \[1^2 = \frac{1(2)(3)}{6} = 1\]

          Inductive step:
          \begin{align*}
            \frac{n(n + 1)(2n + 1)}{6} + (n + 1)^2 & = \frac{(n + 1)(n + 2)(2(n + 1) + 1)}{6} \\
            (2n^2 + n) + 6(n + 1)                  & = (n + 2)(2n + 3)                        \\
                                                   & = 2n^2 + 7n + 6
          \end{align*}
        \end{proof}

  \item Recall that the sequence of Fibonacci numbers is defined as follows: $F_1=F_2=1$, and $F_n=F_{n-1}+F_{n-2}$ when $n>2$.
        \begin{enumerate}
          \item Prove that $F_1^2+F_2^2+\ldots+F_n^2=F_nF_{n+1}$ for all $n\in\mathbb{N}$.
                \begin{proof} Base step: $1^2 = 1 * 1$

                  Inductive step:
                  \begin{align*}
                    F_nF_{n + 1} + F_{n + 1}^2 & = F_{n + 1}F_{n + 2} \\
                    F_n + F_{n + 1}            & = F_{n + 2}
                  \end{align*}

                  This is the definition of $F_n$.
                \end{proof}

          \item Let $S_n$, where $n\in\mathbb{N}$, be the set all $n$-digit binary strings that have no consecutive 1s. Prove that $S_n$ has exactly $F_n+2$ elements.
                \begin{proof} Base step: \\
                  $S_1 = \{0, 1\}$ and $F_3 = 2$. \\
                  $S_2 = \{0, 1, 10\}$ and $F_4 = 3$.

                  Inductive step:
                  
                  Take $S_{n-1}$, and prepend 0 to all of those strings. We will call this set of strings $O_{n-1}$. $O_{n-1}$ has $F_{n+1}$ elements, same as $S_{n-1}$. All elements of $O_{n-1}$ are of length n, and adding only 0's to the strings cannot produce consecutive 1's, therefore $O_{n-1} \subset S_n$.

                  Next, take all of the elements of $S_{n-2}$ and prepend 10 to all of them. We will call this set of strings $T_{n-2}$. $T_{n-2}$ has $F_{n+2}$ elements, same as $S_{n-2}$. All elements of $T_{n-2}$ are of length n. Adding 10 cannot create consecutive 1's, as there is a 0 between our introduced 1 and the first digit of the existing string.
                  
                  Let $O_{n-1} \cup T_{n-2} = U_n$. There cannot be any common elements between $O_{n-1}$ and $T_{n-2}$, because the elements of $O_{n-1}$ all start with 0 and the elements of $T_{n-2}$ all start with 1. This shows that $U_n \subset S_n$ and that $U_n$ has $F_n + F_{n+1} = F_{n+2}$ elements.

                  The next step is to show that $U_n = S_n$. Assume there exists some $s$ such that $s\in S_n$ and $s\notin U_n$. Let $s_i$ be digit $i$ of $s$. $s_1$, the first digit of $s$, must be 0 or 1.

                  Given $s_1 = 0$, the substring $s_2s...s_n$ is a string of length $n - 1$ that has no repeating 0s, so it must be an element of $S_{n-1}$ and therefore $s$ is in $U_n$.

                  Given $s_1 = 1$, $s_2 = 0$ or there would be consecutive 1's. The substring $s_3s...s_n$ must be in $S_{n-2}$, and therefore $s$ is in $U_n$.
                \end{proof}
        \end{enumerate}
  \item Suppose $n$ straight lines lie on a plane in such a way that no two of the lines are parallel and no three of the lines intersect at a single point. Prove that such a collection of lines divides the plane into $\displaystyle\frac{n^2+n+2}{2}$ regions.
        \begin{proof}
          Base step: 1 line seperates a plane into $\displaystyle\frac{1^2+1+2}{2} = 2$ regions.

          Inductive step: When drawing the $n^{\text{th}}$ line it intersects each other line at a different point, as the problem's conditions state. This means there are $n - 1$ intersections when drawing line $n$. This new line creates $n$ new region, because at each intersection it finishes spliting a region into two sections, and then after the last intersection it procedes into the last existing region spliting it into two.

          It is also worth noting that $\displaystyle\frac{n^2+n+2}{2} = \frac{n^2+n}{2} + 1 = \sum_0^n + 1$.

          If we assume the conjecture is true up to $n-1$, we get the following.
          \begin{align*}
            \sum_0^{n-1} + 1 + n & = \sum_0^n + 1 \\
            \sum_0^{n-1} + n     & = \sum_0^n
          \end{align*}
          It becomes clear that these two summations are equal, so the conjecture must be true.
        \end{proof}
\end{enumerate}

\end{document}