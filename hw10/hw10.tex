\documentclass[12pt,a4paper,reqno,parskip=full]{amsart}
\usepackage{amsmath}
\usepackage{amsfonts}
\usepackage{amssymb}
\usepackage{parskip}

\begingroup
\makeatletter
\@for\theoremstyle:=definition,remark,plain\do{
\expandafter\g@addto@macro\csname th@\theoremstyle\endcsname{
\addtolength\thm@preskip\parskip}}
\endgroup

\numberwithin{equation}{section}
\addtolength{\textwidth}{3 truecm}
\addtolength{\textheight}{1 truecm}
\setlength{\voffset}{-.6 truecm}
\setlength{\hoffset}{-1.3 truecm}
\theoremstyle{plain}

\newtheorem{theorem}[subsection]{Theorem}
\newtheorem{proposition}[subsection]{Proposition}
\newtheorem{lemma}[subsection]{Lemma}
\newtheorem{corollary}[subsection]{Corollary}
\newtheorem{claim}[subsection]{Claim}
\newtheorem{conjecture}[subsection]{Conjecture}
\newtheorem{question}[subsection]{Question}
\newtheorem{remark}[subsection]{Remark}
\newtheorem{definition}[subsection]{Definition}
\newtheorem{example}[subsection]{Example}

\theoremstyle{definition}

\renewcommand{\leq}{\leqslant}
\renewcommand{\geq}{\geqslant}
\newcommand{\eps}{\varepsilon}

\DeclareMathOperator{\Aut}{Aut}
\DeclareMathOperator{\modulo}{mod}
\DeclareMathOperator{\End}{End}
\DeclareMathOperator{\Id}{Id}

\def\AA{{\mathcal A}}
\def\CC{{\mathcal C}}
\def\DD{{\mathcal D}}
\def\E{{\mathbb E}}
\def\EE{{\mathcal E}}
\def\FF{{\mathbb F}}
\def\II{{\mathcal I}}
\def\N{{\mathbb N}}
\def\OO{{\mathcal O}}
\def\PP{{\mathcal P}}
\def\Q{{\mathbb Q}}
\def\R{{\mathbb R}}
\def\S{{\mathbb S}}
\def\SS{{\mathcal S}}
\def\UU{{\mathcal U}}
\def\Z{{\mathbb Z}}

\begin{document}

\title{Homework 10}

\author{E. Taber McFarlin}

\maketitle

\begin{enumerate}
  \item Recall that if $G=\{g_1,\ldots,g_n\}$ is a finite group, then the
        multiplication table of $G$ is obtained by filling in the table below
        with all possible products of pairs of elements of $G$.

        Prove that the elements in each row of the multiplication table are a
        permutation of the elements of $G$. Is the same true of the columns?
        What if G is infinite?

        \begin{proof}

        \end{proof}

  \item Let G be a finite group such that $|G|$ is even. Prove that there exists
        an element $g\in G$ that is not equal to the identity $e$ but that
        satisfies $g\cdot g=e$

  \item Recall that a group $G$ is \emph{Abelian} if it is commutative.
        \begin{itemize}
          \item Prove that if $G$ is an \emph{Abelian} group and
                $H=\{a^2\mid a\in G\}$, then $H$ is a subgroup of $G$.
          \item Prove that if $G$ is an \emph{Abelian} group and
                $H=\{a\in G\mid a^2=e\}$, then $H$ is a subgroup of $G$.
        \end{itemize}

  \item Problem 4:

  \item Let $G$ be a group, and let $H$ be a subgroup of $G$. Recall that
        $G/H=\{gH\mid g\in G\}$ is the set of (left) cosets of $H$ in $G$.
        Prove that if LaTeX: G G  is Abelian, then the rule $gH\cdot g'H=gg'H$
        defines a group operation on $G/H$.
\end{enumerate}


\end{document}