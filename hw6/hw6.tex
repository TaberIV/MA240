\documentclass[12pt,a4paper,reqno,parskip=full]{amsart}
\usepackage{amsmath}
\usepackage{amsfonts}
\usepackage{amssymb}
\usepackage{parskip}

\begingroup
\makeatletter
\@for\theoremstyle:=definition,remark,plain\do{
\expandafter\g@addto@macro\csname th@\theoremstyle\endcsname{
\addtolength\thm@preskip\parskip}}
\endgroup

\numberwithin{equation}{section}
\addtolength{\textwidth}{3 truecm}
\addtolength{\textheight}{1 truecm}
\setlength{\voffset}{-.6 truecm}
\setlength{\hoffset}{-1.3 truecm}
\theoremstyle{plain}

\newtheorem{theorem}[subsection]{Theorem}
\newtheorem{proposition}[subsection]{Proposition}
\newtheorem{lemma}[subsection]{Lemma}
\newtheorem{corollary}[subsection]{Corollary}
\newtheorem{claim}[subsection]{Claim}
\newtheorem{conjecture}[subsection]{Conjecture}
\newtheorem{question}[subsection]{Question}
\newtheorem{remark}[subsection]{Remark}
\newtheorem{definition}[subsection]{Definition}
\newtheorem{example}[subsection]{Example}

\theoremstyle{definition}

\renewcommand{\leq}{\leqslant}
\renewcommand{\geq}{\geqslant}
\newcommand{\eps}{\varepsilon}

\DeclareMathOperator{\Aut}{Aut}
\DeclareMathOperator{\modulo}{mod}
\DeclareMathOperator{\End}{End}
\DeclareMathOperator{\Id}{Id}

\def\AA{{\mathcal A}}
\def\CC{{\mathcal C}}
\def\DD{{\mathcal D}}
\def\E{{\mathbb E}}
\def\EE{{\mathcal E}}
\def\FF{{\mathbb F}}
\def\II{{\mathcal I}}
\def\N{{\mathbb N}}
\def\OO{{\mathcal O}}
\def\PP{{\mathcal P}}
\def\Q{{\mathbb Q}}
\def\R{{\mathbb R}}
\def\S{{\mathbb S}}
\def\SS{{\mathcal S}}
\def\UU{{\mathcal U}}
\def\Z{{\mathbb Z}}

\begin{document}

\title{Homework 6}

\author{E. Taber McFarlin}

\maketitle

\begin{enumerate}
  \item Prove or disprove the following statement: For every $n\in\mathbb{N}$, there exist $n$ natural numbers whose sum equals their product.
  \item Prove or disprove the following statement: If $x$ and $y$ are irrational numbers such that $x < y$, then there exists a rational number $z$ such that $x<z<y$.
  \item Prove or disprove the following statement: If $A$, $B$, $C$, and $D$ are sets, then $(A\times B)\cup(C\times D)=(A\cup C)\times(B\cup D)$.
  \item Make a conjecture about which numbers $n\in\mathbb{N}$ can be expressed as a sum of two or more consecutive natural numbers. (Note that the numbers in the sum don't have to start at 1. For example, 12 is such a number since $12=3+4+5$). Then prove your conjecture.
\end{enumerate}

\end{document}