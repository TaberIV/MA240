\documentclass[12pt,a4paper,reqno,parskip=full]{amsart}
\usepackage{amsmath}
\usepackage{amsfonts}
\usepackage{amssymb}
\usepackage{parskip}

\begingroup
\makeatletter
\@for\theoremstyle:=definition,remark,plain\do{
\expandafter\g@addto@macro\csname th@\theoremstyle\endcsname{
\addtolength\thm@preskip\parskip}}
\endgroup

\numberwithin{equation}{section}
\addtolength{\textwidth}{3 truecm}
\addtolength{\textheight}{1 truecm}
\setlength{\voffset}{-.6 truecm}
\setlength{\hoffset}{-1.3 truecm}
\theoremstyle{plain}

\newtheorem{theorem}[subsection]{Theorem}
\newtheorem{proposition}[subsection]{Proposition}
\newtheorem{lemma}[subsection]{Lemma}
\newtheorem{corollary}[subsection]{Corollary}
\newtheorem{claim}[subsection]{Claim}
\newtheorem{conjecture}[subsection]{Conjecture}
\newtheorem{question}[subsection]{Question}
\newtheorem{remark}[subsection]{Remark}
\newtheorem{definition}[subsection]{Definition}
\newtheorem{example}[subsection]{Example}

\theoremstyle{definition}

\renewcommand{\leq}{\leqslant}
\renewcommand{\geq}{\geqslant}
\newcommand{\eps}{\varepsilon}

\DeclareMathOperator{\Aut}{Aut}
\DeclareMathOperator{\modulo}{mod}
\DeclareMathOperator{\End}{End}
\DeclareMathOperator{\Id}{Id}

\def\AA{{\mathcal A}}
\def\CC{{\mathcal C}}
\def\DD{{\mathcal D}}
\def\E{{\mathbb E}}
\def\EE{{\mathcal E}}
\def\FF{{\mathbb F}}
\def\II{{\mathcal I}}
\def\N{{\mathbb N}}
\def\OO{{\mathcal O}}
\def\PP{{\mathcal P}}
\def\Q{{\mathbb Q}}
\def\R{{\mathbb R}}
\def\S{{\mathbb S}}
\def\SS{{\mathcal S}}
\def\UU{{\mathcal U}}
\def\Z{{\mathbb Z}}

\begin{document}

\title{Homework 6}

\author{E. Taber McFarlin}

\maketitle

\begin{enumerate}
  \item Prove or disprove the following statement: For every $n\in\mathbb{N}$, there exist $n$ natural numbers whose sum equals their product.
        \begin{proof}
          Let $x$ and $y$ be natural numbers whose sum equals their product.
          \begin{align*}
            xy       & = x + y           \\
            xy - y   & = x               \\
            y(x - 1) & = x               \\
            y        & = \frac{x}{x - 1}
          \end{align*}
          All pairs of real numbers that satisfy the above function will be two numbers whose sum equals their product. However, for both members of the pair to be natural numbers, it must be true that $x$ is a natural number, and $(x - 1 )\mid x$. This is only true for $x=2$, so the only pair is $x=2, y = 2$. This shows that the conjecture is false.
        \end{proof}
  \item Prove or disprove the following statement: If $x$ and $y$ are irrational numbers such that $x < y$, then there exists a rational number $z$ such that $x<z<y$.
        \begin{proof} Since $x < y$ it is true that $y - x > 0$. \\
          Let $c\in\mathbb{N}$ such that $cy - cx > 1$. There must be some $n\in\mathbb{N}$ such that $cx<n<cy$. This means $\displaystyle x<\frac{n}{c}<y$.
        \end{proof}
  \item Prove or disprove the following statement: If $A$, $B$, $C$, and $D$ are sets, then $(A\times B)\cup(C\times D)=(A\cup C)\times(B\cup D)$.
        \begin{proof}
          If two sets are equal they must have the same number of elements.
          \[|(A\times B)\cup(C\times D)| = |(A\cup C)\times(B\cup D)|\]

          Assume $A$, $B$, $C$, and $D$ are all non-empty have no duplicate elements between them.
          \[|(A\times B)\cup(C\times D)| = |A| * |B| + |C| * |D|\]
          \begin{align*}
            |(A\cup C)\times(B\cup D)| & = (|A| + |C|) * (|B| + |D|)                     \\
                                       & = |A| * |B| + |C| * |D| + |A| * |D| + |B| * |C| \\
          \end{align*}
          \begin{align*}
            |A| * |B| + |C| * |D| & = |A| * |B| + |C| * |D| + |A| * |D| + |B| * |C| \\
            0                     & = |A| * |D| + |B| * |C|
          \end{align*}
          Clearly, the conjecture does not hold in this case.
        \end{proof}
  \item Make a conjecture about which numbers $n\in\mathbb{N}$ can be expressed as a sum of two or more consecutive natural numbers. (Note that the numbers in the sum don't have to start at 1. For example, 12 is such a number since $12=3+4+5$). Then prove your conjecture.
        \conjecture Any $n\in\mathbb{N}$ can be expressed as a sum of two or more consecutive natural numbers if and only if it is not a power of 2.
        \begin{proof}
          $n$ is not a power of 2 $\implies$
          n can be expressed as a sum of two or more consecutive natural numbers. \\
          -
          
          $n$ n can be expressed as a sum of two or more consecutive natural numbers
          $\implies$ is not a power of 2. \\
          -
        \end{proof}
\end{enumerate}

\end{document}