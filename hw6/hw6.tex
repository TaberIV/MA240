\documentclass[12pt,a4paper,reqno,parskip=full]{amsart}
\usepackage{amsmath}
\usepackage{amsfonts}
\usepackage{amssymb}
\usepackage{parskip}

\begingroup
\makeatletter
\@for\theoremstyle:=definition,remark,plain\do{
\expandafter\g@addto@macro\csname th@\theoremstyle\endcsname{
\addtolength\thm@preskip\parskip}}
\endgroup

\numberwithin{equation}{section}
\addtolength{\textwidth}{3 truecm}
\addtolength{\textheight}{1 truecm}
\setlength{\voffset}{-.6 truecm}
\setlength{\hoffset}{-1.3 truecm}
\theoremstyle{plain}

\newtheorem{theorem}[subsection]{Theorem}
\newtheorem{proposition}[subsection]{Proposition}
\newtheorem{lemma}[subsection]{Lemma}
\newtheorem{corollary}[subsection]{Corollary}
\newtheorem{claim}[subsection]{Claim}
\newtheorem{conjecture}{Conjecture}
\newtheorem{question}[subsection]{Question}
\newtheorem{remark}[subsection]{Remark}
\newtheorem{definition}[subsection]{Definition}
\newtheorem{example}[subsection]{Example}

\theoremstyle{definition}

\renewcommand{\leq}{\leqslant}
\renewcommand{\geq}{\geqslant}
\newcommand{\eps}{\varepsilon}

\DeclareMathOperator{\Aut}{Aut}
\DeclareMathOperator{\modulo}{mod}
\DeclareMathOperator{\End}{End}
\DeclareMathOperator{\Id}{Id}

\def\AA{{\mathcal A}}
\def\CC{{\mathcal C}}
\def\DD{{\mathcal D}}
\def\E{{\mathbb E}}
\def\EE{{\mathcal E}}
\def\FF{{\mathbb F}}
\def\II{{\mathcal I}}
\def\N{{\mathbb N}}
\def\OO{{\mathcal O}}
\def\PP{{\mathcal P}}
\def\Q{{\mathbb Q}}
\def\R{{\mathbb R}}
\def\S{{\mathbb S}}
\def\SS{{\mathcal S}}
\def\UU{{\mathcal U}}
\def\Z{{\mathbb Z}}

\begin{document}

\title{Homework 6}

\author{E. Taber McFarlin}

\maketitle

\begin{enumerate}
  \item Prove or disprove the following statement: For every $n\in\mathbb{N}$, there exist $n$
        natural numbers whose sum equals their product.
        \begin{proof}
          If $n = 1$ the case is vaccuously true for any one number. If $n = 2$ there exists the
          example $2 \times 2 = 2 + 2$. For $n > 2$ we can use the set of numbers $n,\,2$ and $n-2$
          ones.
          \begin{align*}
            n \times 2 \times 1 \times \dots \times 1 & = n \times 2 \times 1^{n-2} = 2n \\
            n + 2 + 1 + \dots + 1                     & = n + 2 + (n - 2) = 2n
          \end{align*}
        \end{proof}
  \item Prove or disprove the following statement: If $x$ and $y$ are irrational numbers such that
        $x < y$, then there exists a rational number $z$ such that $x<z<y$.
        \begin{proof} Since $x < y$ it is true that $y - x > 0$. \\
          Let $c\in\mathbb{N}$ such that $cy - cx > 1$. There must be some $n\in\mathbb{N}$ such
          that $cx<n<cy$. This means $\displaystyle x<\frac{n}{c}<y$. $\displaystyle z=\frac{n}{c}$
        \end{proof}
  \item Prove or disprove the following statement: If $A$, $B$, $C$, and $D$ are sets, then
        $(A\times B)\cup(C\times D)=(A\cup C)\times(B\cup D)$.
        \begin{proof}
          If two sets are equal they must have the same number of elements.
          \[|(A\times B)\cup(C\times D)| = |(A\cup C)\times(B\cup D)|\]

          Assume $A$, $B$, $C$, and $D$ are all non-empty have no duplicate elements between them.
          \[|(A\times B)\cup(C\times D)| = |A| * |B| + |C| * |D|\]
          \begin{align*}
            |(A\cup C)\times(B\cup D)| & = (|A| + |C|) * (|B| + |D|)                     \\
                                       & = |A| * |B| + |C| * |D| + |A| * |D| + |B| * |C| \\
          \end{align*}
          \begin{align*}
            |A| * |B| + |C| * |D| & = |A| * |B| + |C| * |D| + |A| * |D| + |B| * |C| \\
            0                     & = |A| * |D| + |B| * |C|
          \end{align*}
          Clearly, the conjecture does not hold in this case.
        \end{proof}
  \item Make a conjecture about which numbers $n\in\mathbb{N}$ can be expressed as a sum of two or
        more consecutive natural numbers. (Note that the numbers in the sum don't have to start at
        1. For example, 12 is such a number since $12=3+4+5$). Then prove your conjecture.
        \conjecture Any $n\in\mathbb{N}$ can be expressed as a sum of two or more consecutive
        natural numbers if and only if it is not a power of 2.
        \begin{proof}
          $n$ is not a power of 2 $\implies$
          n can be expressed as a sum of two or more consecutive natural numbers. \\
          Any $n$ that is not a power of 2 can be represented as: $2^m\times (2k + 1)$, where
          $m\in\mathbb{Z}_{>0}$ and $k\in\mathbb{N}$. We can use these two factors in different
          cases to construct a sequnce of consecutive integers that sums to $n$.

          When $2^{m + 1} > 2k + 1$: \\
          Let $p\in\mathbb{N}$ and $2p + 1 = $ the least factor of $2k + 1$ that is not 1.
          \[n = \left(2^m - \frac{p-1}{2}\right) +\dots 2^m+\dots \left(2^m + \frac{p-1}{2}\right)\]
          We know that $\displaystyle \left(2^m - \frac{2^m}{2}\right)$ will be a natural number
          because \[\frac{p}{2} < \frac{2k + 1}{2} < 2^m.\]

          When $2^{m + 1} < 2k + 1$:
          This case is simplest when $m = 0$, meaning the number is odd.
          \[2k + 1 = k + (k + 1)\]
          This gives us two consecutive natural numbers that sum to $n$. When $m > 0$, we can find
          our consecutive natural numbers by adding to both sides of the sequence. We add a digit to
          both sides until we have $2^{m+1}$ numbers. When adding a digit to both sides we do not
          change the average of the entire sequence, meaning the average is
          $\displaystyle\frac{2k + 1}{2}$.
          Therefore our sum must be
          \[\frac{2k + 1}{2} \cdot 2^{m+1} = {2k + 1} \cdot 2^m = n.\]
          We know that there are at least $m$ natural numbers less than or equal to $k$ because
          \begin{align*}
            2k + 1          & > 2^{m+1} \\
            k + \frac{1}{2} & > 2^m     \\
            k               & \geq 2^m
          \end{align*}

          $n$ is a power of 2 $\implies$
          $n$ cannot be expressed as a sum of two or more consecutive natural numbers. \\
          Assume some sequence of $i$ consecutive numbers exists, and Let $n_i$ represent them.
          \[n_1 + n_2 +\dots n_i = n\]
          Let $m$ be the average of the numbers in this sequence, and thus also $m \cdot i = n$.

          If $i$ is odd then $m$ will be an integer, the number in the middle of the sequence. This 
          would mean an integer times an odd number would be a power of two, which is impossible.

          If $i$ is even then $m$ will not be an integer, and this means two times a non-integer
          would be a power of two. This implies that two does not divide this power of two, which
          is a contradiction.
        \end{proof}
\end{enumerate}

\end{document}