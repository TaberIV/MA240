\documentclass[12pt,a4paper,reqno,parskip=full]{amsart}
\usepackage{amsmath}
\usepackage{amsfonts}
\usepackage{amssymb}
\usepackage{parskip}

\begingroup
\makeatletter
\@for\theoremstyle:=definition,remark,plain\do{
\expandafter\g@addto@macro\csname th@\theoremstyle\endcsname{
\addtolength\thm@preskip\parskip}}
\endgroup

\numberwithin{equation}{section}
\addtolength{\textwidth}{3 truecm}
\addtolength{\textheight}{1 truecm}
\setlength{\voffset}{-.6 truecm}
\setlength{\hoffset}{-1.3 truecm}
\theoremstyle{plain}

\newtheorem{theorem}[subsection]{Theorem}
\newtheorem{proposition}[subsection]{Proposition}
\newtheorem{lemma}[subsection]{Lemma}
\newtheorem{corollary}[subsection]{Corollary}
\newtheorem{claim}[subsection]{Claim}
\newtheorem{conjecture}[subsection]{Conjecture}
\newtheorem{question}[subsection]{Question}
\newtheorem{remark}[subsection]{Remark}
\newtheorem{definition}[subsection]{Definition}
\newtheorem{example}[subsection]{Example}

\theoremstyle{definition}

\renewcommand{\leq}{\leqslant}
\renewcommand{\geq}{\geqslant}
\newcommand{\eps}{\varepsilon}

\DeclareMathOperator{\Aut}{Aut}
\DeclareMathOperator{\modulo}{mod}
\DeclareMathOperator{\End}{End}
\DeclareMathOperator{\Id}{Id}

\def\AA{{\mathcal A}}
\def\CC{{\mathcal C}}
\def\DD{{\mathcal D}}
\def\E{{\mathbb E}}
\def\EE{{\mathcal E}}
\def\FF{{\mathbb F}}
\def\II{{\mathcal I}}
\def\N{{\mathbb N}}
\def\OO{{\mathcal O}}
\def\PP{{\mathcal P}}
\def\Q{{\mathbb Q}}
\def\R{{\mathbb R}}
\def\S{{\mathbb S}}
\def\SS{{\mathcal S}}
\def\UU{{\mathcal U}}
\def\Z{{\mathbb Z}}

\begin{document}

\title{Homework 8}

\author{E. Taber McFarlin}

\maketitle

\begin{enumerate}
  \item Prove that the function $f:\mathbb{N} \times\mathbb{N} \to\mathbb{N}$ defined as
        $f(m,n)=2^{n-1}(2m-1)$ is a bijection.

        \begin{proof} Let $m,n,a,b\in\mathbb{N}$. Assume $f(m,n) = f(a,b)$.

          Injective:
          \begin{align*}
            2^{n - 1}(2m - 1) & = 2^{b - 1}(2a - 1) \\
            2^{n - b}(2m - 1) & = 2a - 1
          \end{align*}
          If $n \neq b$ then $2^{n - b}$ is either a power of two or the inverse of such. This would
          imply that $2a - 1$ either can be divided by a power of two, or can divide a power of two.
          Clearly this is not possible, and it must be true that $n = b$. Then it is true that
          $1(2m - 1) = 2a - 1$ and $m = a$, so the function is injective.

          Surjective: Let $k\in\mathbb{N}$
          \[2^{n - 1}(2m - 1) = k\]
          Any natural number can be expressed in this form. If $k$ is odd then $n = 1$ and
          $2m - 1 = k$. Some power of two times an odd number is proficient to represent any even
          $k$, because it is just the largest power of two $k$ is divisible by times an odd factor.
        \end{proof}
  \item Let $\mathbb{D}=\{(x,y)\in\mathbb{R}^2\mid x^2+y^2\leq 1\}$ be the closed unit disk. Prove
        that $|\mathbb{D}| = |\mathbb{R}^2|$.
        \begin{proof}
          $|\mathbb{D}| \leq |\mathbb{R}^2|$

          $|\mathbb{R}^2| \leq |\mathbb{D}|$

          Consider the set $\mathbb{E} = \{(x,y)\in\mathbb{R}^2\mid x^2+y^2\geq 1\}$.

          Any point in $\mathbb{D}$ can be mapped to a single point in $\mathbb{E}$ and the reverse
          is true as well. Any point in $\mathbb{D}$ or $\mathbb{E}$, $(x, y)$, can be represented
          as a vector with a magnitude of $x^2+y^2$ and a direction of
          $\displaystyle\cos\left(\frac{x}{x^2+y^2}\right)$. If $(x, y)\in\mathbb{E}$ then the point
          with magnitude $\displaystyle\frac{1}{x^2+y^2}$ and direction
          $\displaystyle\cos\left(\frac{x}{x^2+y^2}\right)$ is in $\mathbb{D}$, and vice versa.

          This shows that $|\mathbb{E}| = |\mathbb{D}|$.
        \end{proof}
  \item Prove that there are infinitely many primes of the form $3q + 2$.
        \begin{proof}
          All primes other than 3 are be able to be represented as either $3q + 1$ or $3q + 2$,
          where $q\in\mathbb{N}$, otherwise they would be a multiple of 3 and therefore not prime.

          Assume we have a finite set $P$ of all primes that can be represented as $3q + 2$.
          \[P = {p_1, p_2, \dots, p_n}\]
          Construct a number $N$ such that
          \begin{align*}
            N & = 3p_1p_2 \dots p_n - 1       \\
              & = 3(p_1p_2 \dots p_n - 1) + 2
          \end{align*}
          If $N$ is a prime then it is of the form $3q + 2$, but is not containted in our list.

          If $N$ is composite then it cannot have any prime factors contained in our list $P$.
          Therefore, all of $N$'s prime factors must be of the form $3q + 1$. However, a product of
          primes of form $3q + 1$ will also have the from $3q + 1$, so $N$ cannot exist, and there
          must be infinitely many primes with the form $3q + 3$.
        \end{proof}
\end{enumerate}
\end{document}