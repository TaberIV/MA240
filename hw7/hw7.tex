\documentclass[12pt,a4paper,reqno,parskip=full]{amsart}
\usepackage{amsmath}
\usepackage{amsfonts}
\usepackage{amssymb}
\usepackage{parskip}
\usepackage{enumerate}

\begingroup
\makeatletter
\@for\theoremstyle:=definition,remark,plain\do{
\expandafter\g@addto@macro\csname th@\theoremstyle\endcsname{
\addtolength\thm@preskip\parskip}}
\endgroup

\numberwithin{equation}{section}
\addtolength{\textwidth}{3 truecm}
\addtolength{\textheight}{1 truecm}
\setlength{\voffset}{-.6 truecm}
\setlength{\hoffset}{-1.3 truecm}
\theoremstyle{plain}

\newtheorem{theorem}[subsection]{Theorem}
\newtheorem{proposition}[subsection]{Proposition}
\newtheorem{lemma}[subsection]{Lemma}
\newtheorem{corollary}[subsection]{Corollary}
\newtheorem{claim}[subsection]{Claim}
\newtheorem{conjecture}[subsection]{Conjecture}
\newtheorem{question}[subsection]{Question}
\newtheorem{remark}[subsection]{Remark}
\newtheorem{definition}[subsection]{Definition}
\newtheorem{example}[subsection]{Example}

\theoremstyle{definition}

\renewcommand{\leq}{\leqslant}
\renewcommand{\geq}{\geqslant}
\newcommand{\eps}{\varepsilon}

\DeclareMathOperator{\Aut}{Aut}
\DeclareMathOperator{\modulo}{mod}
\DeclareMathOperator{\End}{End}
\DeclareMathOperator{\Id}{Id}

\def\AA{{\mathcal A}}
\def\CC{{\mathcal C}}
\def\DD{{\mathcal D}}
\def\E{{\mathbb E}}
\def\EE{{\mathcal E}}
\def\FF{{\mathbb F}}
\def\II{{\mathcal I}}
\def\N{{\mathbb N}}
\def\OO{{\mathcal O}}
\def\PP{{\mathcal P}}
\def\Q{{\mathbb Q}}
\def\R{{\mathbb R}}
\def\S{{\mathbb S}}
\def\SS{{\mathcal S}}
\def\UU{{\mathcal U}}
\def\Z{{\mathbb Z}}

\begin{document}

\title{Homework 7}

\author{E. Taber McFarlin}

\maketitle

\begin{enumerate}
  \item Consider the relation $E$ defined on $\mathbb{R} \times\mathbb{R}$ by
        $\left((a,b),(c,d)\right)\in E$ if $|a|+|b| = |c|+|d|$.
        \begin{itemize}
          \item Prove that $E$ is an equivalence relation.
                \begin{proof}
                  For $E$ to be an equivalence relation it must satisfy three conditions:
                  \begin{enumerate}[i]
                    \item Reflexive: $((a,b), (a,b))\in E$
                          \[|a| + |b| = |a| + |b|\]
                    \item Symmetric: $((a,b), (c,d))\in E \implies ((c,d), (a,b))\in E$
                          \[|a| + |b| = |c| + |d|\]
                          \[|c| + |d| = |a| + |b|\]
                    \item Transitive:
                          $((a,b), (c,d)),\,((c,d), (e,f))\in E \implies ((a,b), (e,f))\in E$
                          \[|a| + |b| = |c| + |d| = |e| + |f|\]
                  \end{enumerate}
                \end{proof}
          \item Give a geometric description of the equivalence classes of $E$. \\
                The elements in relation $E$ are pairs of pairs of real numbers. The sum of the
                two real numbers' distance from zero in each pair are equal.
        \end{itemize}
  \item Prove that the relation $R$ on $\mathbb{Z}$ given by
        $(a,b)\in R$ if $2a+3b\equiv0\,\,(\text{mod}\,\,5)$ is an equivalence relation.
        \begin{proof}
          For $R$ to be an equivalence relation it must satisfy three conditions:
          \begin{enumerate}[i]
            \item Reflexive: $(a,a)\in E$
                  \begin{align*}
                    2a + 3a \equiv 0\,\,(\text{mod}\,\,5)
                     & \implies\exists n\in\mathbb{Z}: 5n\mid (2a + 3a) \\
                    5a      \equiv 0\,\,(\text{mod}\,\,5)
                     & \implies 5n\mid 5a                               \\
                     & n  = a
                  \end{align*}
            \item Symmetric: $(a,b)\in E \implies (b, a)\in E$
                  \[5a + 5b \equiv 0\,\,(\text{mod}\,\,5)\]
                  Two multiples of 5 added together will always be congruent modulo five, so we can
                  subtract it from a quantity without changing that quantity's modularity to five.
                  \begin{align*}
                    2a + 3b \equiv 0\,\,(\text{mod}\,\,5)
                                    & \implies\exists n\in\mathbb{Z}: 5n\mid (2a + 3b)  \\
                    -3a - 2b \equiv 0\,\,(\text{mod}\,\,5)
                                    & \implies\exists m\in\mathbb{Z}: 5m\mid -(3a + 2b) \\
                    5(-m) = 3a + 2b & \implies 3a + 2b \equiv 0\,\,(\text{mod}\,\,5)
                  \end{align*}
            \item Transitive: $(a,b),\,(b,c)\in E \implies (a, c)\in E$
                  \begin{align*}
                    2a + 3b \equiv 0\,\,(\text{mod}\,\,5)
                     & \implies\exists n\in\mathbb{Z}: 5n\mid (2a + 3b) \\
                    2b + 3c \equiv 0\,\,(\text{mod}\,\,5)
                     & \implies\exists m\in\mathbb{Z}: 5m\mid (2b + 3c) \\
                    2a + 5b + 3c \equiv 0\,\,(\text{mod}\,\,5)
                     & \implies 5(n + m)\mid (2a + 5b + 3c)
                  \end{align*}
                  Because $5b$ is a multiple of five it does not affect the modulo of the number.
                  Therefore,
                  \begin{align*}
                    5(n + m)\mid (2a + 5b + 3c) & \implies 5(n + m)\mid (2a + 3c)                \\
                                                & \implies 2a + 3c \equiv 0\,\,(\text{mod}\,\,5)
                  \end{align*}
          \end{enumerate}

        \end{proof}
  \item Let $R$ and $S$ be equivalence relations on a set $A$.
        \begin{itemize}
          \item Is $R\cup S$ an equivalence relation on $A$? Back up your answer with a proof.
                \begin{proof}
                  $R\cup S$ is not neccesarily transitive. \\
                  If $R$ contains $(a,b)$ and not $(b,c)$, and $S$ contains
                  $(b,c)$ $R\cup S$ and not $(a, b)$ then $R\cup S$ will contain
                  $(a,b),(b,c)$ and not $(a,c)$.

                  An example of this is: \\
                  $A=\mathbb{Z}$,
                  $(a,b)\in R$ if $a \equiv b\,\,(\text{mod}\,\,5)$,
                  $(a,b)\in S$ if $a \equiv b\,\,(\text{mod}\,\,7)$
                  \[(5,35)\in R,\,(35,7)\in S,\,(5,7)\notin R\cup S\]
                \end{proof}
          \item Is $R\cap S$ an equivalence relation on $A$? Back up your answer with a proof.
                \begin{proof} $R\cap S$ holds all of the properties of an equivalence relation.
                  \begin{enumerate}[i]
                    \item Reflexive: $\forall a\in A,\,(a,a)\in R\cap S$. \\
                          Both $R$ and $S$ are equivalence relations that must contain $(a,a)$, so
                          $(a,a)\in R\cap S$.
                    \item Symmetric: $(a, b)\in R\cap S,\implies (b, a)\in R\cap S$ \\
                          If $(a,b)\in R\cap S$, then $(a,b)$ must be in both $R$ and $S$, which are
                          both equivalence relations so they must both contain $(b,a)$, and 
                          $(b,a)\in(a,a)\in R\cap S$.
                    \item Transitive: $(a,b),\,(b,c)\in R\cap S\implies (a,c)\in R\cap S$ \\
                          If $(a,b)$ and $(b,c)$ are in $R\cap S$, then they must be in both $R$ and
                          $S$. Since $R$ and $S$ are equivalence relations they must also contain
                          $(a,c)$, so $(a,c)\in R\cap S$.
                  \end{enumerate}
                \end{proof}
        \end{itemize}
  \item For elements $[a],[b]\in\mathbb{Z}_n$, recall that  $[a]-[b]=[a-b]$. Now suppose that
        $H\subseteq\mathbb{Z}_n$ is a subset containing $d$ elements and that the relation $R$ on
        $\mathbb{Z}_n$ defined by $([a],[b])\in R$ if $[a]-[b]\in H$ is an equivalence relation.
Prove that the equivalence class (with respect to $R$) of each element of $\mathbb{Z}_n$ has
        $d$ elements, and prove that $d\mid n$.
        \begin{proof}
          
        \end{proof}
\end{enumerate}
\end{document}