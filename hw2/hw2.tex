\documentclass[12pt,a4paper,reqno,parskip=full]{amsart}
\usepackage{amsmath}
\usepackage{amsfonts}
\usepackage{amssymb}
\usepackage{parskip}

\begingroup
\makeatletter
\@for\theoremstyle:=definition,remark,plain\do{
\expandafter\g@addto@macro\csname th@\theoremstyle\endcsname{
\addtolength\thm@preskip\parskip}}
\endgroup

\numberwithin{equation}{section}
\addtolength{\textwidth}{3 truecm}
\addtolength{\textheight}{1 truecm}
\setlength{\voffset}{-.6 truecm}
\setlength{\hoffset}{-1.3 truecm}
\theoremstyle{plain}

\newtheorem{theorem}[subsection]{Theorem}
\newtheorem{proposition}[subsection]{Proposition}
\newtheorem{lemma}[subsection]{Lemma}
\newtheorem{corollary}[subsection]{Corollary}
\newtheorem{claim}[subsection]{Claim}
\newtheorem{conjecture}[subsection]{Conjecture}
\newtheorem{question}[subsection]{Question}
\newtheorem{remark}[subsection]{Remark}
\newtheorem{definition}[subsection]{Definition}
\newtheorem{example}[subsection]{Example}

\theoremstyle{definition}

\renewcommand{\leq}{\leqslant}
\renewcommand{\geq}{\geqslant}
\newcommand{\eps}{\varepsilon}

\DeclareMathOperator{\Aut}{Aut}
\DeclareMathOperator{\modulo}{mod}
\DeclareMathOperator{\End}{End}
\DeclareMathOperator{\Id}{Id}

\def\AA{{\mathcal A}}
\def\CC{{\mathcal C}}
\def\DD{{\mathcal D}}
\def\E{{\mathbb E}}
\def\EE{{\mathcal E}}
\def\FF{{\mathbb F}}
\def\II{{\mathcal I}}
\def\N{{\mathbb N}}
\def\OO{{\mathcal O}}
\def\PP{{\mathcal P}}
\def\Q{{\mathbb Q}}
\def\R{{\mathbb R}}
\def\S{{\mathbb S}}
\def\SS{{\mathcal S}}
\def\UU{{\mathcal U}}
\def\Z{{\mathbb Z}}

\begin{document}

\title{Homework 2}

\author{E. Taber McFarlin}

\maketitle

I pledge my honor that I have abided by the Stevens Honor System.

\begin{enumerate}
  \item{
        \begin{claim}
          Let $r\in\mathbb{Q}_{>0}$. Prove that if $\frac{r^2+1}{r}\leq1$, then $\frac{r^2+2}{r}\leq2$.
        \end{claim}

        Below is a trivial proof showing that $\frac{r^2+1}{r}>1$ for all $r > 0$.
        \begin{proof}

          \begin{align*}
            \frac{r^2 + 1}{r} & = \frac{n}{m}
          \end{align*}

          $n$ and $m$ must be positive real numbers given $r$ is a positive real number, and $\mathbb{Q}_{>0}$ is closed under division.
        \end{proof}}

  \item {
        \begin{claim}
          Prove that if $a$, $b$, and $c$ are odd integers such that $a+b+c=0$, then $abc<0$.
        \end{claim}

        Below is a vacuous proof showing $a+b+c=0$ cannot be true when $a$, $b$, and $c$ are odd.
        \begin{proof}
          Let $e, f, g \in \mathbb{Z}$:
          \begin{align*}
            (2e + 1) + (2f + 1) + (2g + 1) & = 0  \\
            2e + 2f + 2g + 3               & = 0  \\
            2(e + f + g)                   & = -3
          \end{align*}

          Negative 3 is not an even number and therefore cannot be expressed as $2k$ for any integer $k$. This proves that the sum of three odd numbers cannot be 0, and that the above claim is vacuously proven false.
        \end{proof}}

  \item {
        \begin{claim}
          Let $n\in\mathbb{Z}$. Prove that $2n^2+n$ is odd if and only if $\cos\left(\frac{n\pi}{2}\right)$ is even.
        \end{claim}

        \begin{proof}
          There are two relevant cases for $2n^2+n$: $n$ is even or $n$ is odd.

          If $n$ is even: \[2(2k)^2 + 2k = 2((2k)^2 + k)\] so the result is even.

          If $n$ is odd:
          \[
            2(2k + 1)^2 + 2k + 1 = 2((2k + 1)^2 + k) + 1
          \]
          so the result is odd.

          Next, will inspect $\cos\left(\frac{n\pi}{2}\right)$ in the cases when $n$ is even or odd.

          If $n$ is even:
          \[
            \cos\left(\frac{2k\pi}{2}\right) = \cos\left(k\pi\right) = \pm 1,
          \]
          so the result is odd.

          If $n$ is odd:
          \[
            \cos\left(\frac{(2k + 1)\pi}{2}\right) = 0
          \]
          so the result is even.

          
          So far we have proven the following:

          $2n^2+n$ is odd $\iff n$ is odd.

          $\cos\left(\frac{n\pi}{2}\right)$ is even $\iff n$ is odd.

          If these two statments are equivalent to $n$ is odd, then they are equivalent to eachother.
        \end{proof}}

  \item {
        \begin{itemize}
          \item {
                In general, proving that a statement $P$ is true if and only if a statement $Q$  is true requires proving the implication $P\implies Q$ and the implication $Q\implies P$. If both implications are true, then the statements $P$ and $Q$ are said to be \textit{equivalent}.

                Suppose you want to prove that the statements $P$, $Q$, $R$, and $S$ are all equivalent to one another. Explain what you would have to do to complete a proof.}

          \item {
                In order to prove such a statement I would prove $P \implies Q$, then $Q \implies R$, then $R \implies S$, and finally, $S \implies P$.

                This works because if we can prove all of the above statments then we have also proven statments like $Q \implies P$, because $ Q \implies R \implies S \implies P$.
                }
        \end{itemize}
        }

\end{enumerate}


\end{document}