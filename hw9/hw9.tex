\documentclass[12pt,a4paper,reqno,parskip=full]{amsart}
\usepackage{amsmath}
\usepackage{amsfonts}
\usepackage{amssymb}
\usepackage{parskip}

\begingroup
\makeatletter
\@for\theoremstyle:=definition,remark,plain\do{
\expandafter\g@addto@macro\csname th@\theoremstyle\endcsname{
\addtolength\thm@preskip\parskip}}
\endgroup

\numberwithin{equation}{section}
\addtolength{\textwidth}{3 truecm}
\addtolength{\textheight}{1 truecm}
\setlength{\voffset}{-.6 truecm}
\setlength{\hoffset}{-1.3 truecm}
\theoremstyle{plain}

\newtheorem{theorem}[subsection]{Theorem}
\newtheorem{proposition}[subsection]{Proposition}
\newtheorem{lemma}[subsection]{Lemma}
\newtheorem{corollary}[subsection]{Corollary}
\newtheorem{claim}[subsection]{Claim}
\newtheorem{conjecture}[subsection]{Conjecture}
\newtheorem{question}[subsection]{Question}
\newtheorem{remark}[subsection]{Remark}
\newtheorem{definition}[subsection]{Definition}
\newtheorem{example}[subsection]{Example}

\theoremstyle{definition}

\renewcommand{\leq}{\leqslant}
\renewcommand{\geq}{\geqslant}
\newcommand{\eps}{\varepsilon}

\DeclareMathOperator{\Aut}{Aut}
\DeclareMathOperator{\modulo}{mod}
\DeclareMathOperator{\End}{End}
\DeclareMathOperator{\Id}{Id}

\def\AA{{\mathcal A}}
\def\CC{{\mathcal C}}
\def\DD{{\mathcal D}}
\def\E{{\mathbb E}}
\def\EE{{\mathcal E}}
\def\FF{{\mathbb F}}
\def\II{{\mathcal I}}
\def\N{{\mathbb N}}
\def\OO{{\mathcal O}}
\def\PP{{\mathcal P}}
\def\Q{{\mathbb Q}}
\def\R{{\mathbb R}}
\def\S{{\mathbb S}}
\def\SS{{\mathcal S}}
\def\UU{{\mathcal U}}
\def\Z{{\mathbb Z}}

\begin{document}

\title{Homework 9: The Dreaded Calculus}

\author{E. Taber McFarlin}

\maketitle

\begin{enumerate}
  \item Prove that if $f:\mathbb{R}\to\mathbb{R}$ is differentiable at
        $a\in\mathbb{R}$ and $C$ is an arbitrary constant, then the function
        $C\cdot f$ is differentiable at $a\in\mathbb{R}$ as well.

        \begin{proof}
          \begin{align*}
            f'(a)          & = \lim_{h\to 0}\frac{f(a + h) - f(a)}{h}               \\
            (C\cdot f)'(a) & = \lim_{h\to 0}\frac{C\cdot f(a + h) - C\cdot f(a)}{h} \\
                           & = C\left(\lim_{h\to 0}\frac{f(a + h) - f(a)}{h}\right) \\
            C(f'(a))       & = C\cdot f'(a)
          \end{align*}
        \end{proof}
  \item The function $f(x)=e^x$ has the remarkable property that it is equal to
        its own derivative. Do any other functions have this property? Describe
        the set of all functions $f:\mathbb{R} \to\mathbb{R}$ that are equal to
        their own derivatives, and prove that your answer is correct.

        Functions of the form $f(x) = Ce^x$ where $C$ is an arbitrary constant
        are the only functions equal to their own derivative.
  \item Consider the series
        $\displaystyle\sum_{n=1}^\infty\frac{1}{(3n-2)(3n+1)}$
        \begin{itemize}
          \item Compute the first few partial sums $S_N$ of the series and
                conjecture a general formula for $S_N$.

                $n = 1$
                \[\frac{1}{(3 - 2)(3 + 1)} = \frac{1}{(1)(4)} = \frac{1}{4}\]

                $n = 2$
                \[
                  \frac{1}{4} + \frac{1}{(6 - 2)(6 + 1)} =
                  \frac{1}{4} + \frac{1}{28} = \frac{8}{28} = \frac{2}{7}
                \]

                $n = 3$
                \[
                  \frac{2}{7} + \frac{1}{70} =
                  \frac{21}{70} = \frac{3}{10}
                \]

                $n\in\mathbb{N}$
                \[\frac{n}{3n + 1}\]
          \item Use induction to prove that your conjecture is correct.
                \begin{proof}
                  Base step: \emph{See above}

                  Inductive step:
                  \begin{align*}
                    \frac{n - 1}{3n} + \frac{1}{(3n - 2)(3n + 1)} & = 3n \\
                    \frac{n - 1}{3n} + \frac{1}{9n^2 - 3n - 2}    & = 3n
                  \end{align*}
                \end{proof}
          \item Prove that the series converges, and find its sum.
        \end{itemize}
  \item Recall that a function $:\mathbb{R} \to\mathbb{R}$ is continuous at
        $a\in\mathbb{R}$ if $\displaystyle\lim_{x\to a}f(x)=f(a)$, which means
        that for any $\varepsilon>0$ , there exists a $\delta>0$ such
        that $|f(x)-f(a)|<\varepsilon$ whenever $|x-a|<\delta$.

        Let $\mathbb{Q} =\{q_1,\ldots,q_n,\ldots\}$ be an enumeration of the
        rational numbers (recall that $\mathbb{Q}$ is countable!), and consider
        the function $f:\mathbb{R} \to\mathbb{R}$ defined as

        \begin{align*}
          f(x)=
          \begin{cases}
            \frac{1}{n}, & x=q_n              \\
            0,           & x\not\in\mathbb{Q}
          \end{cases}
        \end{align*}

        Prove that $f$ has the remarkable property that it is continuous at
        $a\in\mathbb{R}$ if and only if $a$ is irrational.

        \begin{proof}
          \begin{align*}
            \lim_{x\to a}f(x) = f(a)
          \end{align*}
        \end{proof}
\end{enumerate}

\end{document}