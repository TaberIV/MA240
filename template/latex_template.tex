\documentclass[12pt,a4paper,reqno,parskip=full]{amsart}
\usepackage{amsmath}
\usepackage{amsfonts}
\usepackage{amssymb}
\usepackage{parskip}

\begingroup
\makeatletter
\@for\theoremstyle:=definition,remark,plain\do{
\expandafter\g@addto@macro\csname th@\theoremstyle\endcsname{
\addtolength\thm@preskip\parskip}}
\endgroup

\numberwithin{equation}{section}
\addtolength{\textwidth}{3 truecm}
\addtolength{\textheight}{1 truecm}
\setlength{\voffset}{-.6 truecm}
\setlength{\hoffset}{-1.3 truecm}
\theoremstyle{plain}

\newtheorem{theorem}[subsection]{Theorem}
\newtheorem{proposition}[subsection]{Proposition}
\newtheorem{lemma}[subsection]{Lemma}
\newtheorem{corollary}[subsection]{Corollary}
\newtheorem{claim}[subsection]{Claim}
\newtheorem{conjecture}[subsection]{Conjecture}
\newtheorem{question}[subsection]{Question}
\newtheorem{remark}[subsection]{Remark}
\newtheorem{definition}[subsection]{Definition}
\newtheorem{example}[subsection]{Example}

\theoremstyle{definition}

\renewcommand{\leq}{\leqslant}
\renewcommand{\geq}{\geqslant}
\newcommand{\eps}{\varepsilon}

\DeclareMathOperator{\Aut}{Aut}
\DeclareMathOperator{\modulo}{mod}
\DeclareMathOperator{\End}{End}
\DeclareMathOperator{\Id}{Id}

\def\AA{{\mathcal A}}
\def\CC{{\mathcal C}}
\def\DD{{\mathcal D}}
\def\E{{\mathbb E}}
\def\EE{{\mathcal E}}
\def\FF{{\mathbb F}}
\def\II{{\mathcal I}}
\def\N{{\mathbb N}}
\def\OO{{\mathcal O}}
\def\PP{{\mathcal P}}
\def\Q{{\mathbb Q}}
\def\R{{\mathbb R}}
\def\S{{\mathbb S}}
\def\SS{{\mathcal S}}
\def\UU{{\mathcal U}}
\def\Z{{\mathbb Z}}

\begin{document}

\title{The Title of the Document}

\author{The Author}

\maketitle

\begin{abstract}
A short abstract (if desired).
\end{abstract}

\section{A Section}

Here's some text, which you simply type into your \LaTeX\ file in the obvious way. To write mathematics within a block of text, put dollar signs around the mathematical notation. You might write something like: Consider the function $f(x)=\log_2(x)$, and suppose that $x=2^n$. Here's an equation, set apart from the main text:
\[
x^2+y^2=1.
\]
You can also use various environments to write out more involved computations.

\subsection{A subsection}

Here's a definition:

\begin{definition}
A function $f:\R\to\R$ is \emph{differentiable at $x$} if the limit
\[
f'(x)=\lim_{h\to 0}\frac{f(x+h)-f(x)}{h}
\]
exists.
\end{definition}

And here's a theorem:

\begin{theorem}
If $f$ is a continuous function defined on the closed interval $[a,b]$, and if $f(x)=F'(x)$, then
\[
F(b)-F(a)=\int_a^bf(x)\,dx.
\]
That is, the total change of $F$ over $[a,b]$ is equal to the definite integral of its rate of change over $[a,b]$.
\end{theorem}

\begin{proof}
This is where the proof goes. The white box at the end is meant to signal that the proof is complete.
\end{proof}

For a reference, see \cite{ref}.

\begin{thebibliography}{10}

\bibitem{ref} A reference (which \LaTeX\ will cite in the text using the label you create for it in the bibliography).

\end{thebibliography}

\end{document}