\documentclass[12pt,a4paper,reqno,parskip=full]{amsart}
\usepackage{amsmath}
\usepackage{amsfonts}
\usepackage{amssymb}
\usepackage{parskip}

\begingroup
\makeatletter
\@for\theoremstyle:=definition,remark,plain\do{
\expandafter\g@addto@macro\csname th@\theoremstyle\endcsname{
\addtolength\thm@preskip\parskip}}
\endgroup

\numberwithin{equation}{section}
\addtolength{\textwidth}{3 truecm}
\addtolength{\textheight}{1 truecm}
\setlength{\voffset}{-.6 truecm}
\setlength{\hoffset}{-1.3 truecm}
\theoremstyle{plain}

\newtheorem{theorem}[subsection]{Theorem}
\newtheorem{proposition}[subsection]{Proposition}
\newtheorem{lemma}[subsection]{Lemma}
\newtheorem{corollary}[subsection]{Corollary}
\newtheorem{claim}[subsection]{Claim}
\newtheorem{conjecture}[subsection]{Conjecture}
\newtheorem{question}[subsection]{Question}
\newtheorem{remark}[subsection]{Remark}
\newtheorem{definition}[subsection]{Definition}
\newtheorem{example}[subsection]{Example}

\theoremstyle{definition}

\renewcommand{\leq}{\leqslant}
\renewcommand{\geq}{\geqslant}
\newcommand{\eps}{\varepsilon}

\DeclareMathOperator{\Aut}{Aut}
\DeclareMathOperator{\modulo}{mod}
\DeclareMathOperator{\End}{End}
\DeclareMathOperator{\Id}{Id}

\def\AA{{\mathcal A}}
\def\CC{{\mathcal C}}
\def\DD{{\mathcal D}}
\def\E{{\mathbb E}}
\def\EE{{\mathcal E}}
\def\FF{{\mathbb F}}
\def\II{{\mathcal I}}
\def\N{{\mathbb N}}
\def\OO{{\mathcal O}}
\def\PP{{\mathcal P}}
\def\Q{{\mathbb Q}}
\def\R{{\mathbb R}}
\def\S{{\mathbb S}}
\def\SS{{\mathcal S}}
\def\UU{{\mathcal U}}
\def\Z{{\mathbb Z}}

\begin{document}

\title{MA 240 - Homework 1}

\author{E. Taber McFarlin}

\begin{abstract}
  I prove that the product of an even number and an odd number is even.
\end{abstract}

\maketitle


\section{Introduction}

The concepts involved in this proof are introduced in grade school, and are defined below:

\begin{definition}
  An even number is an integer that is divisible by two.
\end{definition}

\begin{definition}
  An odd number is an integer that is not divisible by two.
\end{definition}

\begin{theorem}
  Each natural number is the product of one unique combination of prime numbers.
\end{theorem}

\begin{proof}
  If we multiply together an even number $(e)$ and odd number $(o)$, we can show this as multiplying the prime factors of each number altogether, including any repeating factors. We'll use $e_i$ and $o_i$ to show the prime factors of $e$ and $o$, respectively. (For now we will assume both $e$ and $o$ are natural numbers.)
  \[
    e * o = n
  \]
  \[
    (e_0 e_1 ... e_n)(o_0 o_1 ... o_m) = n
  \]

  Since $e$ is known to be even, we know that 2 is at least one of its prime factors. Because of this, we can replace $e_0$ with 2.
  \[
    (2 e_1 ... e_n)(o_0 o_1 ... o_m) = n
  \]

  Since all of $e_i$ and $o_i$ are prime, we can show $n$ as:
  \[
    n = 2 p_1 p_2 ... p_n
  \]

  $n$ is the product of 2 and other integers, meaning it is a multiple of two, and $n$ is even.

  Up until now we have assumed both $e$ and $o$ are natural numbers, meaning we have not covered the 0 and negative cases. 
  
  0 is an even number, and $0 * n = 0$ for any $n$, so that case also holds.

  If either, or both, $e$ or $o$ are negative we could simply factor out $-1$ and find the prime factors of the magnitudes of the negative number(s), and the rest of the logic remains consistent.
\end{proof} 
\end{document}